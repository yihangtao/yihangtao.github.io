\documentclass[letterpaper,11pt]{article}




	%Linux Libertine



    
\newlength{\outerbordwidth}
\pagestyle{empty}
\raggedbottom
\raggedright
\usepackage{hyperref}
\usepackage{xcolor}
\hypersetup{
    colorlinks=true,
    linkcolor={red!50!black},
    citecolor={blue!50!black},
    urlcolor={blue!30!black}
}


\usepackage{framed}
%	\usepackage{times}
\usepackage{libertine}
\usepackage{tocloft}
\usepackage{graphicx}
\usepackage{multirow}
\usepackage[utf8]{inputenc}
\usepackage{tabularx}
\usepackage{helvet}
\usepackage{ragged2e}
\usepackage{titlesec}
\title{LYNCH-CV}
%-----------------------------------------------------------
%Edit these values as you see fit

\setlength{\outerbordwidth}{3pt}  % Width of border outside of title bars
\definecolor{shadecolor}{gray}{0.75}  % Outer background color of title bars (0 = black, 1 = white)
\definecolor{shadecolorB}{gray}{0.93}  % Inner background color of title bars

\titleformat{\section}{
  \vspace{-4pt}\scshape\raggedright\large
}{}{0em}{}[\color{black}\titlerule \vspace{-5pt}]
%-----------------------------------------------------------
%Margin setup

\setlength{\evensidemargin}{-0.25in}
\setlength{\headheight}{0in}
\setlength{\headsep}{0in}
\setlength{\oddsidemargin}{-0.25in}
\setlength{\paperheight}{11in}
\setlength{\paperwidth}{8.5in}
\setlength{\tabcolsep}{0in}
\setlength{\textheight}{9.5in}
\setlength{\textwidth}{7in}
\setlength{\topmargin}{-0.3in}
\setlength{\topskip}{0in}
\setlength{\voffset}{0.1in}
\setlength{\headheight}{-10pt} 

%-----------------------------------------------------------
%Custom commands
% \newcommand{\resitem}[1]{\item #1 \vspace{-2pt}}
% \newcommand{\resheading}[1]{\vspace{2pt}
%   \parbox{\textwidth}{\setlength{\FrameSep}{\outerbordwidth}
%     \begin{shaded}
% \setlength{\fboxsep}{0pt}\framebox[\textwidth][l]{\setlength{\fboxsep}{1pt}\fcolorbox{shadecolorB}{shadecolorB}{\textbf{\sffamily{\mbox{~}\makebox[6.842in][l]{\large #1} \vphantom{p\^{E}}}}}}
%     \end{shaded}
%   }\vspace{-5pt}
% }
\newcommand{\ressubheading}[4]{
\begin{tabular*}{6.5in}{l@{\cftdotfill{\cftsecdotsep}\extracolsep{\fill}}r}
		\textbf{#1} & #2 \\
		\textit{#3} & \textit{#4} \\
\end{tabular*}\vspace{-6pt}}
%-----------------------------------------------------------

 %   \usepackage{lmodern} 
% \usepackage[top=0.75in,bottom=0.75in,right=0.75in,left=0.75in,headheight=30pt,headsep=1cm]{geometry}
% \usepackage{fancyhdr}
% \usepackage[english]{babel} % Required to compile in Windows
% \usepackage[letterspace=150]{microtype}
% \usepackage{fontspec}
% %\setlength{\headheight}{110pt} %%or
% \renewcommand{\headrulewidth}{0.4pt}
% \renewcommand{\footrulewidth}{0.4pt}
% %\usepackage[headheight=500pt]{geometry}
% \pagestyle{fancy}
% %\lhead{Robert Lynch CV}
% \fancyhead[L]{Yihang Tao, Curriculum Vitae}
%\fancyfoot[C]{Confidential}
%\fancyfoot[C] {\thepage}
 
\begin{document}

%\chead{Lynch CV}


\moveleft.5\hoffset\centerline{\LARGE\bf YIHANG TAO}


 
\moveleft.5\hoffset\centerline{\large 83 Tat Chee Avenue, Kowloon, Hong Kong}
\moveleft.5\hoffset\centerline{\large(+852) 92856274 $|$ yihang.tommy@my.cityu.edu.hk $|$ \href{https://yihangtao.github.io}{yihangtao.github.io}}
%-----------------------------------------------------------  



\vspace{1mm}
%%%%%%%%%%%%%%%%%%%%%%%%%%%%%%
\section{\textbf{EDUCATION}}
%%%%%%%%%%%%%%%%%%%%%%%%%%%%%%
\vspace{1mm}

\noindent {\href{https://www.cityu.edu.hk/}{\textbf{\large City University of Hong Kong}}\hfill Hong Kong SAR\\
\noindent {PhD in Computer Science\hfill Sep. 2024 - Present\\
\noindent {Supervisor: \href{https://www.cs.cityu.edu.hk/~yugufang/}{Prof. Yuguang "Michael" Fang}}\\
\noindent {JC STEM Lab of Smart City, WINET Laboratory}\\

\vspace{1mm}
\noindent {\href{https://en.sjtu.edu.cn/}{\textbf{\large Shanghai Jiao Tong University}}\hfill Shanghai, China\\
\noindent {Master of Engineering in Electronic Information\hfill Sep. 2021 - Apr. 2024\\
\noindent {Outstanding Graduate (2024)}\\

\vspace{1mm}
\noindent {\href{https://www.seu.edu.cn/}{\textbf{\large Southeast University}}\hfill Nanjing, China\\
\noindent {Bachelor of Engineering in Information Engineering\hfill Sep. 2017 - Jul. 2021\\
\noindent {GPA: 3.89/4.0}\\

%%%%%%%%%%%%%%%%%%%%%%%%%%%%%%
% \resheading{PRESENTATIONS}
\section{\textbf{Research Interests}}
\vspace{2mm}
%%%%%%%%%%%%%%%%%%%%%%%%%%%%%%

\begin{justify}
    \normalsize
    Multi-agent collaborative perception, autonomous driving, AI security, foundation models for multi-agent systems, parameter-efficient fine-tuning.
\end{justify}

% %%%%%%%%%%%%%%%%%%%%%%%%%%%%%%
% \resheading{LANGUAGE}
% %%%%%%%%%%%%%%%%%%%%%%%%%%%%%%
% \center
% \large\textbf{IELTS Overall Score: 7.0}
% \vspace{2mm}

% Listening: 7.0 \qquad \qquad \qquad Reading: 7.5 \qquad \qquad \qquad Writing: 6.5 \qquad \qquad Speaking: 7.0

%%%%%%%%%%%%%%%%%%%%%%%%%%%%%%
\section{\textbf{PUBLICATIONS}}
\vspace{1mm}
% \resheading{PUBLICATIONS}
%%%%%%%%%%%%%%%%%%%%%%%%%%%%%%

\large\textbf{Published \& Accepted:} (\# Equal contribution, * Corresponding author)

\begin{justify}
    \normalsize
    S. Hu, Y. Ma, \textbf{Y. Tao},  Z. Fang, Z. Fang, Y. Deng, S. Kwong, and Y. Fang. ``Task-Aware Parameter-Efficient Fine-Tuning of Large Pre-Trained Models at the Edge,'' \textit{IEEE Global Communications Conference (GLOBECOM)}, Taipei, Taiwan, Dec. 2025.
\end{justify}
\begin{justify}
    \normalsize
    Z. Fang, J. Wang, Y. Ma, \textbf{Y. Tao}, Y. Deng, X. Chen, and Y. Fang. ``R-ACP: Real-Time Adaptive Collaborative Perception Leveraging Robust Task-Oriented Communications,'' \textit{IEEE Journal on Selected Areas in Communications (JSAC)}, 2025.
\end{justify}
\begin{justify}
    \normalsize
    \textbf{Y. Tao}, S. Hu, Z. Fang, and Y. Fang. ``Directed-CP: Directed Collaborative Perception for Connected and Autonomous Vehicles via Proactive Attention,'' \textit{IEEE International Conference on Robotics and Automation (ICRA)}, Atlanta, USA, 2025.
\end{justify}
\begin{justify}
    \normalsize
    S. Hu\#, \textbf{Y. Tao}\#, G. Xu, Y. Deng, X. Chen, Y. Fang, and S. Kwong. ``CP-Guard: Malicious Agent Detection and Defense in Collaborative Bird's Eye View Perception,'' \textit{The 39th Annual AAAI Conference on Artificial Intelligence (AAAI)}, Philadelphia, USA, 2025. \textbf{(Oral Presentation, $<$5\%)}
\end{justify}
\begin{justify}
    \normalsize
    \textbf{Y. Tao}, J. Wu, Q. Pan, A. K. Bashir, and M. Omar. ``O-RAN-Based Digital Twin Function Virtualization for Sustainable IoV Service Response: An Asynchronous Hierarchical Reinforcement Learning Approach,'' \textit{IEEE Transactions on Green Communications and Networking (TGCN)}, vol. 8, no. 3, pp. 1049-1060, Sep. 2024.
\end{justify}
\begin{justify}
    \normalsize
    \textbf{Y. Tao}, J. Wu, X. Lin, S. Mumtaz, and S. Cherkaoui. ``Digital Twin and DRL-Driven Semantic Dissemination for 6G Autonomous Driving Service,'' \textit{IEEE Global Communications Conference (GLOBECOM)}, Kuala Lumpur, Malaysia, Dec. 2023, pp. 2075-2080.
\end{justify}
\begin{justify}
    \normalsize
    \textbf{Y. Tao}, J. Wu, X. Lin, and W. Yang. ``DRL-Driven Digital Twin Function Virtualization for Adaptive Service Response in 6G Networks,'' \textit{IEEE Networking Letters (LNET)}, vol. 5, no. 2, pp. 125-129, Jun. 2023.
\end{justify}

\vspace{2mm}
\large\textbf{Preprints \& Under Review:}

\begin{justify}
    \normalsize
    Z. Fang, Z. Lin, S. Hu, \textbf{Y. Tao}, Y. Deng, X. Chen, and Y. Fang. ``Dynamic Uncertainty-aware Multimodal Fusion for Outdoor Health Monitoring,'' \textit{arXiv preprint arXiv:2508.01001}, 2025.
\end{justify}

\begin{justify}
    \normalsize
    \textbf{Y. Tao}\#, S. Hu\#, Y. Hu, H. An, H. Cao, and Y. Fang. ``GCP: Guarded Collaborative Perception with Spatial-Temporal Aware Malicious Agent Detection,'' \textit{arXiv preprint arXiv:2501.02450}, 2025.
\end{justify}

\begin{justify}
    \normalsize
    \textbf{Y. Tao}, S. Hu, H. An, Z. Fang, H. Cao, and Y. Fang. ``Adaptive Attack on Multi-Agent Collaborative Perception,'' \textit{Under Review}, 2025.
\end{justify}

\begin{justify}
    \normalsize
    S. Hu, \textbf{Y. Tao}, Z. Fang, G. Xu, Y. Deng, S. Kwong, and Y. Fang. ``CP-Guard+: A New Paradigm for Malicious Agent Detection and Defense in Collaborative Perception,'' \textit{arXiv preprint arXiv:2502.07807}, 2025.
\end{justify}

%%%%%%%%%%%%%%%%%%%%%%%%%%%%%%
% \resheading{COMPETITIONS AND AWARDS}
\section{\textbf{HONORS \& AWARDS}}
\vspace{2mm}
%%%%%%%%%%%%%%%%%%%%%%%%%%%%%%

\begin{flushleft}
    \normalsize
    {IEEE Robotics and Automation Society (RAS) Travel Grant for ICRA'25 \cftdotfill{\cftdotsep}Mar. 2025}\\
    \vspace{1mm}
    {Outstanding Graduate, Shanghai Jiao Tong University \cftdotfill{\cftdotsep}May 2024}\\
    \vspace{1mm}
    {WEICHAI POWER Scholarship, Shanghai Jiao Tong University \cftdotfill{\cftdotsep}Oct. 2023}\\
    \vspace{1mm}
    {Excellent League Member, Shanghai Jiao Tong University \cftdotfill{\cftdotsep}Apr. 2022}\\
    \vspace{1mm}
    {National Student Research Training Program Excellence Award (\textbf{Leader}) \cftdotfill{\cftdotsep}Oct. 2020}\\
    \vspace{1mm}
    {Excellence Prize, 2nd International Data Competition, IKCEST (\textbf{top 3\%}) \cftdotfill{\cftdotsep}Oct. 2020}\\
    \vspace{1mm}
    {Sun Qingyun Innovation Scholarship, Southeast University (\textbf{$<$1\% annually}) \cftdotfill{\cftdotsep}Jun. 2020}\\
    \vspace{1mm}
    {Finalist, 36th Mathematical Contest in Modeling (MCM), COMAP (\textbf{top 1\%}) \cftdotfill{\cftdotsep}Apr. 2020}\\
    \vspace{1mm}
    {First Prize, 12th National Information Security Contest, China (\textbf{top 8\%}) \cftdotfill{\cftdotsep}Aug. 2019}\\
    \vspace{1mm}
    {First Prize, 15th Advanced Mathematics Competition, Jiangsu (\textbf{top 10\%}) \cftdotfill{\cftdotsep}Aug. 2018}
\end{flushleft}



%%%%%%%%%%%%%%%%%%%%%%%%%%%%%%
\section{\textbf{PROJECT EXPERIENCE}}
\vspace{2mm}
%%%%%%%%%%%%%%%%%%%%%%%%%%%%%%

\begin{justify}
\normalsize
\textbf{Multi-Agent Collaborative Perception for Autonomous Driving}\cftdotfill{\cftdotsep} 2024 - Present\\
1) Institution and Supervisor: City University of Hong Kong, Prof. Yuguang Fang\\
\vspace{0.5mm}
2) Research Focus: Developing robust collaborative perception systems, designing defense mechanisms against adversarial attacks, and deploying on real-world ROS and Jetson-based autonomous driving platforms;\\
\vspace{0.5mm}
3) Achievements: 3 first-author papers accepted/submitted to top venues (ICRA'25, AAAI'25), 1 RAS travel grant.
\vspace{3.5mm}

\textbf{Digital Twin and 6G Communications}\cftdotfill{\cftdotsep} 2021 - 2024\\
1) Institution and Supervisor: Shanghai Jiao Tong University, Prof. Jun Wu\\
\vspace{0.5mm}
2) Research Focus: Designing digital twin function virtualization for IoV services, developing DRL-based adaptive service response mechanisms, using Unity Software to connect with real-world elevator systems;\\
\vspace{0.5mm}
3) Achievements: 3 first-author papers published in IEEE journals/conferences, Outstanding Graduate Award.
\vspace{3.5mm}

\textbf{Ultrasonic Anti-Recording Security System}\cftdotfill{\cftdotsep} 2019 - 2020\\
1) Institution and Supervisor: Southeast University, Prof. Yubo Song\\
\vspace{0.5mm}
2) Responsibility: Team leader. Designed ultrasonic anti-recording mechanism based on acoustic parametric array; Implemented SM4-based spread spectrum and DDS waveform generation on STM32F407 microcontroller;\\
\vspace{0.5mm}
3) Achievements: First prize in National Information Security Contest, 1 patent (CN111064543A), Sun Qingyun Innovation Scholarship.
\end{justify}

% \begin{justify}
%     \normalsize
%     2012 Lab, Huawei Technologies Ltd. Co. {\textit {``Digital Twin-Driven Internet-Connected Elevator Platform for Anomaly Detection.''}} \textit{Seminar on Digital Twin-based 6G Communications,} March 2022, Shanghai, China. (Presentation)
% \end{justify}
% \resheading{ACADEMIC APPOINTMENT}


\section{\textbf{ACADEMIC SERVICE}}
\vspace{2mm}
%%%%%%%%%%%%%%%%%%%%%%%%%%%%%%

\begin{flushleft}
    \normalsize
    \textbf{Program Committee Member:}\\
    \vspace{1mm}
    {ACM MM 2025, ICML 2025, ICLR 2025, AAAI 2025, ICRA 2025, IUI 2025, IJCNN 2025}\\
    \vspace{1mm}
    {IEEE ISBI 2025, IEEE ICC 2025, IEEE GLOBECOM 2023-2025, IEEE ICCC 2024}\\
    \vspace{2mm}
    \textbf{Journal Reviewer:}\\
    \vspace{1mm}
    {IEEE TMC, IEEE TITS, IEEE TCE, Pattern Recognition, Neural Networks, EAAI, IEEE JBHI, IEEE LNET}\\
    \vspace{2mm}
    \textbf{Session Chair:}\\
    \vspace{1mm}
    {IEEE GLOBECOM 2023, MWN Track, Semantic Communications Session}
\end{flushleft}

%%%%%%%%%%%%%%%%%%%%%%%%%%%%%%
\section{\textbf{TECHNICAL SKILLS}}
\vspace{2mm}
%%%%%%%%%%%%%%%%%%%%%%%%%%%%%%

\begin{flushleft}
    \normalsize
    \textbf{Programming Languages:} Python, MATLAB, C++, C\#, JavaScript, Verilog HDL, \LaTeX, Markdown\\
    \vspace{1mm}
    \textbf{Deep Learning Frameworks:} PyTorch, TensorFlow, OpenMMLab (MMDetection, MMSegmentation)\\
    \vspace{1mm}
    \textbf{Tools \& Platforms:} Git, Docker, Linux, CARLA Simulator, SUMO, ROS\\
    \vspace{1mm}
    \textbf{Languages:} English (IELTS 7.0), Chinese (Native)
\end{flushleft}



\end{document}